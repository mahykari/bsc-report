\begin{abstract}
  علّت‌یابی خطا در سیستم‌های هم‌روند با
  روش‌های دستی و آزمون و خطا فرایند زمان‌بر
  و پیچیده‌ای است. استفاده از روش‌های
  خودکار مانند روش علت واقعی که توسط
  هالپرن و پرل مطرح شده است نیز تنها
  روی سیستم‌های بسیار ساده انجام‌شدنی است.
  ما بهبودهای تئوری زمان اجرا برای مساله علت واقعی
  برای سیستم‌های هم‌روند را بررسی می‌کنیم.
  همچنین، ما ابزار
  \lr{EStResT}
  را برای بررسی علت خطا در یک سیستم هم‌روند
  که با یک ساختمان رویداد توصیف شده است، معرفی می‌کنیم.
  نتایج ما نشان می‌دهند که بهبودهای پیاده‌سازی شده،
  امکان بررسی ساختمان‌هایی با ۲۰ رویداد را
  به راحتی فراهم می‌کنند.
\end{abstract}