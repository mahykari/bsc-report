\documentclass{article}

\usepackage{hyperref}
\usepackage{amsmath,amssymb,amsthm}
\usepackage{enumitem}
\usepackage{cancel}
\usepackage{algorithm}
\usepackage{algpseudocodex}
\usepackage{tikz}
\usepackage{xepersian}

\usetikzlibrary{quotes,angles}

\newtheorem{theorem}{قضیه}

\theoremstyle{definition}
\newtheorem{definition}[theorem]{تعریف}

\newtheorem{proposition}[theorem]{گزاره}

\theoremstyle{remark}
\newtheorem{example}[theorem]{مثال}

\settextfont{Yas}
\setdigitfont{Yas}

\renewcommand{\baselinestretch}{1.35}

\newcommand{\ES}{ \mathcal{E} }
\newcommand{\minen}{ \vdash_{\min} }
\newcommand{\set}[1]{ \left\{ #1 \right\} }

\DeclareMathOperator{\Par}{Par}
\DeclareMathOperator{\Con}{Con}
\DeclareMathOperator{\Min}{Min}
\DeclareMathOperator{\Conf}{\mathcal{F}}
\DeclareMathOperator{\Copy}{\texttt{Copy}}
\DeclareMathOperator{\TopolSort}{\texttt{TopologicalSort}}

\title{استدلال علّی برای خطا در سیستم‌های هم‌روند}

\author{مهیار کریمی}

\begin{document}

\maketitle

\begin{abstract}
  علّت‌یابی خطا در سیستم‌های هم‌روند با
  روش‌های دستی و آزمون و خطا فرایند زمان‌بر
  و پیچیده‌ای است. استفاده از روش‌های
  خودکار مانند روش علت واقعی که توسط
  هالپرن و پرل مطرح شده است، در حالت ساده
  و بدون بهبود نیز تنها
  روی سیستم‌های بسیار ساده انجام‌شدنی است.
  ما ابزار
  \lr{EStResT}
  را برای بررسی علت خطا در یک سیستم هم‌روند
  که با یک ساختمان رویداد توصیف شده است، معرفی می‌کنیم.
  همچنین، بهبودهای تئوری زمان اجرا برای مساله علت واقعی
  در سیستم‌های هم‌روند را بررسی می‌کنیم
  و تسریع نتیجه شده از افزودن آن‌ها به ابزار
  \lr{EStResT}
  را ارائه می‌کنیم.
  نتایج ما نشان می‌دهند که بهبودهای پیاده‌سازی شده،
  امکان بررسی سیستم‌هایی با اندازه قابل قبول
  نسبت به پیچیدگی ذاتی مساله علیت را
  به راحتی فراهم می‌کنند.
\end{abstract}
\section{مقدمه}
سیستم‌های هم‌روند، با توجه به ماهیتشان
و امکان تفکیک بین اجزای کوچکتر یک سیستم،
در بسیاری از شاخه‌های علوم کامپیوتر مورد
استفاده قرار گرفته‌اند؛ برای مثال، یک شبکه
کامپیوتری، یک سیستم هم‌روند گسترده است که
در آن، هر عنصر شبکه به صورت موازی
(و هم‌روند)
با سایر عناصر فعالیت می‌کند. سیستم‌های
هم‌روند دنیای واقعی
(مانند شبکه داخلی یک دانشگاه)
معمولا معماری پیچیده‌ای دارند
و به همین دلیل، مدیریت و علت‌یابی خطا در
آن‌ها نیز مساله دشواری است؛ به همین جهت،
استفاده از روش‌های خودکار علت‌یابی نسبت
به روش‌های آزمون و خطا اولویت می‌یابند.
با توجه به اهمیت این سیستم‌ها، اطمینان از
درستی عملکرد و علت‌یابی خطا در آن‌ها از
اهمیت بالایی برخوردار است.


در درستی‌سنجی صوری سیستم‌ها، یک سیستم و
رفتارهای مورد نظر آن در قالب یک مدل ریاضی
نمایش داده می‌شوند و به کمک روش‌های
الگوریتمی می‌توانیم برقراری رفتار مورد نظر
در سیستم را بررسی کنیم. روش‌های متفاوتی،
مانند ترجمه به اتوماتون محدود، برای مدل
کردن ریاضی سیستم‌های هم‌روند استفاده
می‌شوند. ساختمان رویداد، یک روش برای توصیف
سیستم‌های هم‌روند است که توسط وینسکل در
\cite{winskel1987event,winskel1989introduction}
معرفی شده است. در این ساختمان،
ویژگی‌های اصلی یک سیستم هم‌روند 
(مانند امکان رخ دادن دو رویداد بدون وابستگی به یکدیگر و به صورت موازی)
پوشش داده می‌شوند؛ از این جهت، این روش
برای مدل کردن سیستم‌های مورد نظر ما، روش
مناسبی است.

هالپِرن و پِرل ‌در
\cite{halpern2001causes}
یک فریم‌ورک برای تشخیص علت‌
در یک مدل معادله‌ای ارائه داده‌اند.
این فریم‌ورک، بسیاری از ایده‌های طبیعی علّیت را
(مانند در نظر گرفتن سناریوهایی که در دنیای واقعی صورت نپذیرفته‌اند)
پوشش می‌دهد. از طرفی، با توجه به فرم منظم و
استقرایی ساختمان رویداد، ترجمه یک ساختمان رویداد
به مدل معادله‌ای نیز قابل انجام است
(روشی برای ترجمه توسط صیحانی و دیگران در
\cite{seyhani2022}
ارائه شده است).
به این ترتیب، می‌توانیم از فریم‌ورک تشخیص علت
هالپرن و پرل برای استدلال علّی در
یک سیستم هم‌روند استفاده کنیم.

یکی از مسائل اصلی در ترجمه ساختمان رویداد
و تشخیص علت‌، پیچیدگی زمان اجرای الگوریتم‌های
ارائه شده برای این مراحل است. ترجمه ساختمان رویداد
به مدل معادله‌ای، مدلی با اندازه نمایی 
(نسبت به تعداد رویدادها)
نتیجه می‌دهد. هم‌چنین، روش تشخیص علت ارائه شده
در مدل هالپرن و پرل نیز در کلاس مسائل
\lr{$\Sigma_2^P$-Complete}
قرار می‌گیرد. از این جهت، تشخیص علت در
یک ساختمان رویداد، با کمک تعریف اولیه علیت،
تنها روی سیستم‌های بسیار ساده
با تعداد انگشت‌شماری رویداد قابل انجام است.
هدف این پروژه، بهبود زمان اجرای تشخیص علت
برای میسر کردن بررسی سیستم‌های پیچیده‌تر می‌باشد.

نتیجه اصلی این پروژه، پیاده‌سازی ابزار
\lr{EStResT}
\LTRfootnote{
  \url{https://github.com/seyhani/causality/tree/master/estrest}
}
است که امکان توصیف ساختمان رویداد
و بررسی درستی علت واقعی در آن را فراهم می‌کند.
این ابزار با زبان
\lr{Python}
پیاده‌سازی شده است.

\textbf{ساختار گزارش:}
ابتدا در بخش‌های
\ref{sec:event-structure} و \ref{sec:causality}
مفاهیم ساختمان رویداد، مدل علّی
و علّت واقعی را توضیح می‌دهیم.
در بخش
\ref{sec:es-to-cm}
شیوه ترجمه یک ساختمان رویداد
به مدل علی را توضیح می‌دهیم.
سپس در بخش
\ref{sec:improvements}
بهبودهای تئوری مورد نظر برای مساله
علت واقعی را مورد بررسی قرار می‌دهیم.
در بخش
\ref{sec:experiments}
نتایج عملی را می‌آوریم
و در انتها، در بخش
\ref{sec:future}
گام‌های بعدی برای این پژوهش را مطرح می‌کنیم.

\section{تعریف‌های اولیه و بنیادین}\label{sec:preliminaries}

\section{ساختمان رویداد}\label{sec:event-structure}

روش‌های متفاوتی برای توصیف و مدل کردن
سیستم‌های هم‌روند ارائه شده‌اند.
در توصیف به کمک اتوماتون محدود قطعی
\LTRfootnote{Deterministic finite automaton}
معمولا ترتیب‌های متفاوت اجرای هم‌روند به صورت مستقل
در ساختار مدل توصیف می‌شوند. این روش، ویژگی‌هایی از
یک سیستم هم‌روند واقعی، مانند امکان رخ دادن دو رویداد
به صورت مستقل و بدون وابستگی را فراهم نمی‌کند.
\textit{ساختمان رویداد}
\LTRfootnote{Event structure}
مدلی برای توصیف سیستم‌های هم‌روند است که مشابه
اتوماتون محدود غیرقطعی
\LTRfootnote{Non-deterministic finite automaton}
هم‌روندی رویدادها را به صورت صریح مشخص می‌کند.

\begin{definition}\label{def:event-structure}
  \textit{ساختمان رویداد}
  $\ES$
  یک سه‌تایی به فرم
  $(E, \#, \vdash)$
  است که در آن:
  \begin{enumerate}
    \item $E$
    یک مجموعه از رویداد‌ها است.
    \item $\#$
    رابطه
    \textit{تعارض}
    \LTRfootnote{Conflict}
    است که یک رابطه دودویی متقارن و غیربازتابی بر روی مجموعه
    $E$
    می‌باشد.

    \item $\vdash\;\subseteq \Con(\ES) \times E$
    رابطه
    \textit{فعال‌سازی}
    \LTRfootnote{Enabling}
    است که شرط زیر را برقرار می‌کند:
    \begin{equation*}
        (X \vdash e) \wedge (X \subseteq Y \in \Con(\ES))
        \implies Y \vdash e
    \end{equation*}
  \end{enumerate}
\end{definition}

در رابطه بالا
$\Con$
(مخفف
\lr{Consistent})،
زیرمجموعه‌ای از مجموعه توانی رویدادها است که اعضای آن فاقد تعارض باشند.

به صورت شهودی، رابطه تعارض برای جلوگیری از رخ دادن
دو رویداد در یک اجرا از سیستم،
و رابطه فعال‌سازی برای ایجاد وابستگی بین رویدادها استفاده می‌شود.
برای مشخص کردن مجموعه کمینه وابستگی‌های یک رویداد،
رابطه فعال‌سازی کمینه را تعریف می‌کنیم.

\begin{definition}\label{def:min-enabling}
  به ازای هر ساختمان رویداد، می‌توانیم رابطه
  \textit{فعال‌سازی کمینه}
  را به صورت زیر تعریف کنیم:
  \begin{equation*}
    X \minen e \iff
    (X \vdash e) \wedge
    (\forall Y \subseteq X.\;Y \vdash e \implies Y=X)
  \end{equation*}
\end{definition}

\begin{proposition}
  در هر ساختمان رویداد داریم
  \begin{equation*}
    Y \vdash e \implies
    \exists X \subseteq Y.\;X \minen e
  \end{equation*}
\end{proposition}

برای مشخص کردن وضعیت یک سیستم در هر لحظه
از مفهومی به نام پیکر‌بندی
\LTRfootnote{Configuration}
استفاده می‌شود. یک پیکربندی
مجموعه‌ای شامل رویدادهایی است که تا آن لحظه در سیستم رخ داده‌اند.

\begin{definition}\label{def:configuration}
  اگر
  $\ES = (E,\#,\vdash)$
  یک ساختمان رویداد باشد، یک
  \textit{پیکربندی}
  آن یک زیرمجموعه از رویداد‌ها
  $X \subseteq E$
  است که شرایط زیر را داشته باشد:
  \begin{enumerate}
      \item $X \in \Con(\ES)$
      \item $
        \forall e \in X.\;
        \exists e_0, \cdots, e_n \in X.\;
        e_n = e \wedge
        \forall i \leq n. \set{e_0, \cdots, e_{i-1}} \vdash e_i
      $
  \end{enumerate}
\end{definition}

مجموعه همه پیکربندی‌های یک ساختمان رویداد مانند
$\ES$
با
$\Conf(\ES)$
نمایش داده می‌شود.

\begin{example}\label{ex:event-structure}
  ساختمان رویدادی با مولفه‌های زیر در نظر بگیریم:
  \begin{align*}
    E & = \set{a, b, c} \\
    \# & : a \# b \\
    \vdash & :
      \varnothing \vdash a,\,
      \varnothing \vdash b,\,
      \set{a} \vdash c,\,
      \set{b} \vdash c
  \end{align*}

  شبکه
  \LTRfootnote{Lattice}
  پیکربندی‌های این ساختمان در شکل
  \ref{fig:es-example-configs}
  آمده است.

  \begin{figure}
  \centering
  \def\W{2}
  \def\H{1.25}
  \begin{tikzpicture}[scale=0.8]
      \node (null) at (0,0) {$\varnothing$};
      \node (a) at (-\W,\H) {$\set{a}$};
      \node (b) at (\W,\H)  {$\set{b}$};
      \node (ab) at (0,2*\H) {$\set{a,b}$};
      \node (abc) at (0,3*\H) {$\set{a,b,c}$};
      \draw[->] (null) -- (a);
      \draw[->] (null) -- (b);
      \draw[->] (a) -- (ab);
      \draw[->] (b) -- (ab);
      \draw[->] (ab) -- (abc);
  \end{tikzpicture}
  \caption{
    شبکه پیکربندی‌های مثال
    \ref{ex:event-structure}
  }
  \label{fig:es-example-configs}
\end{figure}

\end{example}
\section{مدل علّی و علّت واقعی}\label{sec:causality}

مساله تشخیص علیت در شاخه‌های مختلف علوم کامپیوتر
(به طور خاص، هوش مصنوعی)
کاربرد دارد. روش‌های متفاوتی برای توصیف علیت ارائه شده‌اند
که به طور خلاصه می‌توانیم آن‌ها را به دو دسته تقسیم کنیم:
روش‌هایی که از مفاهیم منطق مجرد استفاده می‌کنند
(مانند روش
\textit{استدلال غیریکنوای}
گِفنِر در
\cite{geffner1990causal})،
و روش‌هایی که از شبکه‌های بِیزی بدست آمده‌اند
(مانند روش
\textit{علت و اثر}
پرل در
\cite{pearl1999reasoning}).
در روش هالپرن و پرل
(که در دسته دوم قرار می‌گیرد)
تعریف خوبی از علّت واقعی ارائه شده است و
جنبه‌های شهودی اصلی علیت در نظر گرفته شده‌اند.
در ادامه به تعریف مدل علّی
و سپس علیت در این مدل می‌پردازیم.

در ابتدا فرض می‌کنیم که مجموعه‌ای مانند
$X$
از
\textit{متغیرهای تصادفی}
در دست داریم.
مجموعه
(محدود)
مقدارهای ممکن برای هر متغیر مانند
$X_i$
را
\textit{دامنه} $X_i$
می‌نامیم و با
$D(X_i)$
نمایش می‌دهیم
($D(X)$
نیز برای نمایش دامنه مجموعه متغیر
$X$
استفاده می‌شود).
با داشتن مجموعه‌های
$X,Y$
و مقداردهی‌های
$x \in D(X), y \in D(Y)$،
برای نمایش اتحاد
$x$ و $y$
از
$xy$
استفاده می‌کنیم.
با داشتن
$x$،
مقدار یک تک‌متغیر
\LTRfootnote{Singleton}
مانند
$X_i \in X$
را با
$x(X_i)$
نمایش می‌دهیم.

\begin{definition}\label{def:causal-model}
  مدل علّی
  $M$
  یک سه‌تایی به فرم
  $(U,V,F)$
  است که در آن،
  $U$
  مجموعه متغیرهای
  \textit{بیرونی}،
  $V$
  مجموعه متغیرهای
  \textit{درونی}
  و
  $F = \set{F_X \mid X \in V}$
  مجموعه‌ای از توابع به فرم زیر است:
  \[ F_X: D(\Par(X)) \to D(X) \]
  $\Par(X)$
  مجموعه متغیرهای پدر
  $X$
  می‌باشد که مقدار
  $X$
  به مقدار آن‌ها وابسته است.
\end{definition}

\begin{definition}\label{def:causal-world}
  با داشتن مدل علّی
  $M$
  و مقداردهی
  $u$
  برای متغیرهای بیرونی آن،
  زوج
  $(M,u)$
  را یک
  \textit{دنیای علّی}
  می‌نامیم.
\end{definition}

در این گزارش، ما تنها مدل‌های علّی
\textit{بازگشتی}
را بررسی می‌کنیم؛ در مدل‌های بازگشتی،
به ازای یک مقداردهی مانند
$u$
برای متغیرهای بیرونی،
مقدار همه متغیرهای درونی
به صورت یکتا تعیین می‌شود؛
این مقداردهی را با
$Y_M(u)$ (یا اختصارا
$Y(u)$)
نمایش می‌دهیم.

\begin{definition}\label{def:causal-submodel}
  با داشتن مدل علّی
  $M=(U,V,F)$
  و مقداردهی
  $x$
  برای
  $X \in V$،
  \textit{زیرمدل} $M$
  (نسبت به
  $X=x$)
  را با
  $M_{X=x} = (U, V, F_{X=x})$
  نمایش می‌دهیم، که در آن
  \begin{equation*}
    F_{X=x} =
      \set{F_Y \mid Y \in V \setminus X} \cup
      \set{F_{X'} = x(X') \mid X' \in X}
  \end{equation*}
\end{definition}

برای اختصار،
$M_{X=x}$ و $F_{X=x}$
را با
$M_x$ و $F_x$
نمایش می‌دهیم؛ همچنین،
به ازای
$Y \subseteq V$،
$Y_{M_x}(u)$
را به اختصار با
$Y_x(u)$
نمایش می‌دهیم.

\begin{definition}\label{def:causal-network}
  \textit{شبکه علّی}
  برای مدل علّی
  $M=(U,V,F)$،
  به صورت زیر تعریف می‌شود:
  \begin{enumerate}[label=(\alph*)]
    \item به‌ازای هر متغیر در
    $U \cup V$
    یک راس در نظر می‌گیریم.
    \item یال
    $X \to Y$
    را اضافه می‌کنیم اگر و فقط اگر
    مقدار
    $Y$
    به
    $X$
    وابسته باشد
    ($X \in F_Y$).
  \end{enumerate}
\end{definition}

\begin{example}\label{ex:causal-model}
  فرض کنیم دو نفر در یک جنگل هستند
  و می‌خواهند با انداختن یک کبریت روشن،
  جنگل را آتش بزنند.
  دو متغیر بیرونی
  $U_1,U_2$
  را برای نشان دادن قصد نفر اول و دوم
  برای آتش زدن جنگل تعریف می‌کنیم.
  متغیرهای درونی
  $A_1,A_2$
  را برای نشان دادن انداختن کبریت روشن
  توسط نفر اول و دوم در جنگل،
  و متغیر
  $F$
  را نیز برای نشان دادن آتش گرفتن جنگل تعریف می‌کنیم. 

  در این مدل، داریم:
  \begin{align*}
    U & = \set{U_1,U_2} \\
    V & = \set{A_1,A_2,F} \\
    F & :
      F_{A_1}=U_1,\,
      F_{A_2}=U_2,\,
      F_{F}=A_1 \vee A_2
  \end{align*}

  برای این مساله، شبکه علّی به صورت شکل
  \ref{fig:cm-example-network}
  می‌باشد.

  \begin{figure}
  \centering
  \def\W{1.5}
  \def\H{0.8}
  \begin{tikzpicture}
    \node (U1) at (   0, \H) {$U_1$};
    \node (U2) at (   0,-\H) {$U_2$};
    \node (A1) at (  \W, \H) {$A_1$};
    \node (A2) at (  \W,-\H) {$A_2$};
    \node (F)  at (2*\W,  0) {$F$};
    \draw[->] (U1) -- (A1);
    \draw[->] (U2) -- (A2);
    \draw[->] (A1) -- (F);
    \draw[->] (A2) -- (F);
  \end{tikzpicture}
  \caption{
    شبکه علّی برای مثال
    \ref{ex:causal-model}
  }
  \label{fig:cm-example-network}
\end{figure}

\end{example}

با داشتن تعریف‌های بالا،
می‌توانیم مفاهیم علت ضعیف و قوی را توضیح دهیم.

\begin{definition}\label{def:causality}
  با داشتن مدل علّی
  $M=(U,V,F)$،
  مجموعه متغیرهای
  $X$ و $Y$
  و مقداردهی‌های
  $x$ و $y$،
  $X=x$
  یک
  \textit{علت ضعیف}
  برای
  $Y=y$
  در دنیای علّی
  $(M,u)$
  است اگر و فقط اگر شرطهای زیر برقرار باشند:

  \textbf{\lr{AC1}.} $X(u)=x \wedge Y(u)=y$.

  \textbf{\lr{AC2}.}
  مجموعه‌ای مانند
  $W \subseteq V \setminus X$
  و مقداردهی‌های
  $w \in D(W)$ و $\bar{x} \in D(X)$
  وجود داشته باشند، به طوری که:
  \begin{itemize}
    \item[(\lr{a})] $Y_{\bar{x}w} \neq y$.
    \item[(\lr{b})] $Y_{xw}=y$.
    \item[(\lr{c})]
    به ازای هر
    $Z \subseteq V \setminus (X \cup W)$
    به طوری که
    $z = Z(u)$،
    $Y_{xwz}(u)=y$.
  \end{itemize}

  همچنین،
  $X=x$
  یک
  \textit{علت واقعی}
  برای
  $Y=y$
  است اگر و فقط اگر شرط زیر هم برقرار باشد:
  
  \textbf{\lr{AC3}.} $X$
  کمینه باشد؛ در واقع، هیچ زیرمجموعه‌ای از
  $X$
  شرطهای
  \lr{AC1} و \lr{AC2}
  را برآورده نکند.
\end{definition}

در ادبیات علیت، گزاره
$Y=y$
\textit{اثر}
\LTRfootnote{Effect}
نامیده می‌شود.
در ادامه این گزارش، فرض می‌کنیم اثر مورد نظر
یک تک‌متغیر است. این فرض در راستای استفاده ما
از مفهوم علیت است.

\begin{example}\label{ex:causality}
  در سناریوی آتش گرفتن جنگل که در مثال
  \ref{ex:causal-model}
  ارائه شد، با فرض
  $u=(1,1)$،
  $A_1=1$
  علت واقعی
  $F=1$
  است: هر دو گزاره
  $A_1(u)=1$ و $F(u)=1$
  برقرار هستند.
  با تغییر مقدار
  $A_1$ $(\bar{x})$ و $A_2$ $(w)$
  به ۰
  داریم
  $F_{\bar{x}w}(u)=0$.
  واضح است که با بازگرداندن مقدار
  $A_1$ به ۱
  همیشه مقدار
  $F$
  برابر ۱ باقی می‌ماند.
  بنابراین
  $A_1=1$
  دو شرط اول تعریف
  \ref{ex:causality}
  را برآورده می‌کند و یک علت ضعیف است؛
  به راحتی می‌توان نشان داد که این مقداردهی
  شرط سوم را نیز برقرار می‌کند،
  و در نتیجه یک علت قوی است.
  به طور مشابه،
  $A_2=1$
  نیز یک علت قوی است.
\end{example}

\textbf{پیچیدگی زمانی مساله علیت:}
آیتر و لوکاسویچ در
\cite{eiter2001complexity}
نشان داده‌اند که با داشتن
$X=x$ و $Y=y$،
مساله
\textit{تصمیم}
علیت
(با علت
$X=x$
و اثر
$Y=y$)
در کلاس پیچیدگی
\lr{$\Sigma_2^P$-Complete}
قرار می‌گیرد. منابع اصلی این پیچیدگی،
جستجو برای
$W$
و بررسی به‌ازای همه مقادیر
$Z$
در شرط دوم، و همچنین بررسی شرط سوم
در تعریف
\ref{def:causality}
می‌باشد.

\section{ترجمه ساختمان رویداد به مدل علّی}\label{sec:es-to-cm}

برای این‌که بتوانیم از تعریف علیت هالپرن و پرل
برای یک ساختمان رویداد استفاده کنیم، بایستی ابتدا
ساختمان رویداد را به صورت یک مدل علّی بیان کنیم. در ادامه
روشی که توسط صیحانی و دیگران در
\cite{seyhani2022}
برای این ترجمه ارائه شده را توضیح می‌دهیم.

\begin{definition}
  مجموعه $Y$،
  مجموعه $X$
  را \textit{می‌پوشاند}
  اگر و فقط اگر شرط زیر برقرار باشد:
  \begin{equation*}
    X \subset Y \wedge \left(
      \forall Z.\;
      X \subseteq Z \subseteq Y \implies Z=X \vee Z=Y 
    \right)
  \end{equation*}
  رابطه پوشیده شدن
  $X$ با $Y$
  را به صورت
  $X \prec Y$
  نشان می‌دهیم.
\end{definition}

ساختمان رویداد
$\ES=(E,\#,\vdash)$
را در نظر بگیریم، که در آن
$E=\set{e_1,\cdots,e_n}$.
در ادامه مدل علّی
$M=(U,V,F)$
را برای این ساختمان تعریف می‌کنیم.

مجموعه $U$
را برابر مجموعه تهی در نظر می‌گیریم.
برای مجموعه $V$
متغیرهای زیر را در نظر می‌گیریم:
\begin{align}
  V = & \set{C_{e_i,e_j} \mid 1 \leq i < j \leq n} \cup \\
  & \set{
    EN_{S,e} \mid
    S \subseteq E \wedge e \in E \wedge e \not\in S
  } \cup \\
  & \set{
    M_{S,e} \mid
    S \subseteq E \wedge e \in E \wedge e \not\in S
  } \cup \set{PV}
\end{align}

متغیرهای
$C$، $EN$ و $M$
به ترتیب روابط تعارض، فعال‌سازی و فعال‌سازی کمینه
را مدل می‌کنند. همچنین، متغیر
$PV$
برای نشان دادن رفتار ناامن در ساختمان رویداد
(مطابق عبارت \ref{eq:unsafe-behavior})
استفاده می‌شود که در بخش
\ref{sec:improvements}
راجع به آن بیشتر توضیح خواهیم داد.
برای متغیرهای تعارض، داریم:
\begin{equation*}
  F_{C_{e,e'}} = \begin{cases}
    \top  & e \# e' \\
    \bot  & e \cancel{\#} e'
  \end{cases}
\end{equation*}

برای تابع سایر متغیرها، دو تابع کمکی زیر را تعریف می‌کنیم:
\begin{align*}
  \Con(S) & =
  \bigwedge_{e_i,e_j \in S \wedge i < j} \neg C_{e_i,e_j} \\
  \Min(S,e) & = \bigwedge_{
    S' \subseteq E.\;
    (S' \prec S \vee S \prec S') \wedge e \not\in S'
  } \neg M_{S',e}
\end{align*}

به این ترتیب، داریم:
\begin{equation*}
  F_{M_{S,e}} = \begin{cases}
    \Min(S,e) \wedge \Con(S) & S \minen e \\
    \bot                     & S \not\minen e
  \end{cases}
\end{equation*}

\begin{equation*}
  F_{EN_{S,e}} = \left(
    M_{S,e} \vee
    \left( \bigvee_{S' \prec S} EN_{S',e} \right)
  \right) \wedge \Con(S)
\end{equation*}

\section{بهبود پیچیدگی زمانی استدلال علّی}\label{sec:improvements}

\subsection{علت واقعی تک‌متغیره}
با توجه به تعریف مساله علت واقعی،
برای علت‌های چندمتغیره، بررسی شرط کمینه بودن
اندازه علت معادل حل کردن مساله علت واقعی به ازای
همه زیرمجموعه‌های علت ارائه شده می‌باشد.
بنابراین، یکی از راه‌های کاهش پیچیدگی مساله علت واقعی،
ساده‌تر شدن بررسی شرط کمینه بودن است.
قضیه زیر که توسط آیتر و لوکاسویچ در
\cite{eiter2001complexity}
ارائه شده است، به ما در ساده کردن
بررسی شرط کمینه بودن کمک می‌کند.

\begin{theorem}\label{thm:causality-singleton}
  با داشتن
  $(M,u)$،
  $X$ و $Y$،
  اگر
  $X=x$
  علت واقعی
  $Y=y$ باشد،
  $X$
  یک تک‌متغیر است.
\end{theorem}

\begin{proof}[اثبات اولیه]
  با استفاده از برهان خلف و نشان دادن این‌که همیشه
  می‌توانیم همه متغیرهای
  $X$
  (بجز یک متغیر)
  را به مجموعه
  $W$
  منتقل کنیم، این قضیه ثابت خواهد شد.
  اثبات کاملتر توسط آیتر و لوکاسویچ ارائه شده است.
\end{proof}

به کمک این قضیه، می‌توانیم بررسی شرط کمینه بودن را
با پیچیدگی زمانی ثابت انجام دهیم.
توجه کنیم که این قضیه به تنهایی پیچیدگی کلی
مساله علیت را کاهش نمی‌دهد و همچنان این مساله در کلاس
\lr{$\Sigma_2^P$-Complete}
قرار دارد.

\subsection{کاهش اندازه مدل علّی}

دوتا از منابع پیچیدگی در مساله تصمیم علیت،
به دلیل جستجو در مجموعه توانی متغیرهای مدل علّی هستند.
به همین دلیل، می‌توانیم تلاش کنیم تا اندازه مدل علّی را
تا حد ممکن کوچک کنیم. مفهوم
\textit{تصویر}
مدل علّی که توسط هاپکینز در
\cite{hopkins2002strategies}
گامی در جهت حذف تعدادی از متغیرها در
یک نمونه
\LTRfootnote{Instance}
از مساله تصمیم علیت است.

\begin{definition}\label{def:variable-deletion}
  فرض کنیم
  $M=(U,V,F)$
  یک مدل علّی و
  $(M,u)$
  یک دنیای علّی است
  $(V={V_1,\cdots,V_n})$.
  \textit{حذف کردن}
  متغیر
  $V_i$
  از
  $(M,u)$
  را به صورت زیر تعریف می‌کنیم:
  \begin{enumerate}[label=(\alph*)]
    \item $V_i$
    را از
    $V$
    حذف می‌کنیم.
    \item
    به ازای هر فرزند
    $V_i$
    مانند
    $X$،
    در
    $F_X$
    به‌جای
    $V_i$،
    مقدار
    $v_i=V_i(u)$
    را جایگزین می‌کنیم.
  \end{enumerate}
\end{definition}

به صورت شهودی، حذف کردن یک متغیر مانند
$V$
از مدل علّی، معادل ثابت نگه داشتن مقدار آن
(با مقداردهی $V(u)$)
می‌باشد.

\begin{definition}\label{def:projection}
  \textit{تصویر}\LTRfootnote{Projection} $(M,u)$
  روی متغیرهای
  $V_1,\cdots,V_k$
  را به صورت یک مدل علّی جدید تعریف می‌کنیم
  که در آن، متغیرهای
  $V_{k+1},\cdots,V_n$
  از
  $(M,u)$
  حذف شده‌اند.
\end{definition}

\begin{definition}\label{def:w-projection}
  برای
  $(M,u)$
  \textit{تصویر $W$}\LTRfootnote{W-projection}
  به‌ازای مساله علیت
  $X$ و $Y$
  را به صورت تصویر
  $(M,u)$
  روی متغیرهای زیر تعریف می‌کنیم:
  \begin{enumerate}[label=(\alph*)]
    \item $X,Y$
    \item هر راسی مانند
    $V^{XY}$
    که در شبکه علّی مورد نظر،
    روی مسیری از
    $X$ به $Y$
    قرار دارد.
    \item $\Par(V)$
    به ازای هر
    $V$
    که در مجموعه
    $V^{XY}$ها
    و $Y$
    قرار دارد.
  \end{enumerate}
\end{definition}

در ادامه می‌توانیم نشان دهیم که
برای حل کردن مساله تصمیم علیت برای
$X$ و $Y$،
تنها کافی‌است متغیرهایی را در نظر بگیریم که
روی مسیری از
$X$ به $Y$
قرار دارند، یا روی مقدار این متغیرها
تاثیر مستقیم دارند.
قضیه زیر که توسط هاپکینز ارائه شده است،
از مفهوم
تصویر $W$
برای حل مساله علیت استفاده می‌کند.

\begin{theorem}\label{thm:causality-w-projection}
  فرض کنیم
  $M=(U,V,F)$
  یک مدل علّی و
  $(M,u)$
  یک دنیای علّی است.
  $X=x$
  علت واقعی
  $Y=y$
  در
  $(M,u)$
  است، اگر و فقط اگر در
  $(M',u)$
  نیز علت واقعی باشد و
  $M'$
  هم
  تصویر $W$ی
  دنیای علی
  $(M,u)$
  (نسبت به
  $X$ و $Y$)
  باشد.
\end{theorem}

به این ترتیب، حل کردن مساله علیت در مدلی که
به کمک
تصویر $W$
کوچک شده، معادل حل کردن آن در مدل اولیه است.
کاهش اندازه مدل
(و به دست آوردن تصویر $W$)
در زمان چندجمله‌ای قابل انجام است،
و زمان مورد نیاز برای حل مساله علیت در مدل کوچک‌شده
نیز به صورت  قابل توجهی
(با توجه به پیچیدگی مساله)
کاهش می‌یابد.

\textbf{تعریف متغیر اثر در رفتار ناامن ساختمان رویداد:}
برای استفاده از تکنیک
تصویر $W$
در مدل علّی ساختمان رویداد،
بایستی متغیری را به عنوان متغیر اثر تعریف کنیم.
درستی این متغیر، معادل
امکان برقراری حداقل یکی از پیکربندی‌های ناامن می‌باشد.
در قدم اول، رابطه‌ای تعریف می‌کنیم که معادل
امکان رخ دادن یک پیکربندی در ساختمان رویداد است.

\begin{theorem}\label{th:conf-equiv}
  با داشتن یک ساختمان رویداد
  $\ES=(E,\#,\vdash)$
  و مجموعه
  $S = \set{e_1,\cdots,e_n} \subseteq E$،
  داریم
  \begin{equation}\label{eq:conf-equiv}
    S \in \Conf(\ES) \iff
    S \in \Con(E)
    \wedge \exists \pi \in \Pi(S).\;
    \varnothing \vdash \pi_1 \wedge \cdots \wedge
    \set{\pi_1,\cdots,\pi_{n-1}} \vdash \pi_n
  \end{equation}
  که
  $\Pi(S)$
  مجموعه همه جایشگت‌های رویدادهای
  $S$
  است.
\end{theorem}

\begin{proof}
  این قضیه را در هر دو جهت اثبات می‌کنیم.
  \begin{itemize}
    \item $\implies$:
    این گزاره به صورت مستقیم از تعریف
    \ref{def:configuration}
    نتیجه می‌شود.
    \item $\impliedby$:
    از فرض مساله می‌توان نتیجه گرفت که حداقل
    یک مسیر در شبکه پیکربندی‌های
    $\ES$
    از مجموعه تهی
    ($\varnothing$)
    تا $S$
    وجود دارد؛ بنابر این،
    $S$
    یک پیکربندی
    $\ES$
    است و حکم اثبات می‌شود.
  \end{itemize}
\end{proof}

به این ترتیب، می‌توانیم از عبارتی معادل
امکان رخ دادن یک پیکربندی در مدل علّی ساختمان رویداد
استفاده کنیم. توجه کنیم که گزاره
\ref{eq:conf-equiv}
معادل گزاره زیر است که با متغیرهای مدل علّی
ساختمان رویداد بیان شده است:
\begin{equation}\label{eq:conf-equiv-cm}
  S \in \Conf(\ES) \iff
  \Con(S) \wedge
  \bigvee_{\pi \in \Pi(S)} \left(
    EN_{\varnothing, \pi_1} \wedge \cdots \wedge
    EN_{\set{\pi_1,\cdots,\pi_{n-1}}, \pi_n}
  \right)
\end{equation}

به این ترتیب، توصیف رفتار ناامن نیز به صورت زیر قابل انجام است:

\begin{multline}\label{eq:unsafe-behavior-equiv-cm}
  \exists S' \in S.\;S'\in \Conf(\ES) \iff \\
  \bigvee_{S' \in S} \left(
    \Con(S') \wedge
    \bigvee_{\pi \in \Pi(S')} \left(
      EN_{\varnothing, \pi_1} \wedge \cdots \wedge
      EN_{\set{\pi_1,\cdots,\pi_{n-1}}, \pi_n}
    \right)
  \right)
\end{multline}

\section{نتایج عملی}\label{sec:experiments}

\bibliographystyle{plain-fa}
\bibliography{references.bib}
\end{document}