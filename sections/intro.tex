\section{مقدمه}
سیستم‌های هم‌روند، با توجه به ماهیتشان
و امکان تفکیک بین اجزای کوچکتر یک سیستم،
در بسیاری از شاخه‌های علوم کامپیوتر مورد
استفاده قرار گرفته‌اند؛ برای مثال، یک شبکه
کامپیوتری، یک سیستم هم‌روند گسترده است که
در آن، هر عنصر شبکه به صورت موازی
(و هم‌روند)
با سایر عناصر فعالیت می‌کند. سیستم‌های
هم‌روند دنیای واقعی
(مانند شبکه داخلی یک دانشگاه)
معمولا معماری پیچیده‌ای دارند
و به همین دلیل، مدیریت و علت‌یابی خطا در
آن‌ها نیز مساله دشواری است؛ به همین جهت،
استفاده از روش‌های خودکار علت‌یابی نسبت
به روش‌های آزمون و خطا اولویت می‌یابند.
با توجه به اهمیت این سیستم‌ها، اطمینان از
درستی عملکرد و علت‌یابی خطا در آن‌ها از
اهمیت بالایی برخوردار است.


در درستی‌سنجی صوری سیستم‌ها، یک سیستم و
رفتارهای مورد نظر آن در قالب یک مدل ریاضی
نمایش داده می‌شوند و به کمک روش‌های
الگوریتمی می‌توانیم برقراری رفتار مورد نظر
در سیستم را بررسی کنیم. روش‌های متفاوتی،
مانند ترجمه به اتوماتون محدود، برای مدل
کردن ریاضی سیستم‌های هم‌روند استفاده
می‌شوند. ساختمان رویداد، یک روش برای توصیف
سیستم‌های هم‌روند است که توسط وینسکل در
\cite{winskel1987event,winskel1989introduction}
معرفی شده است. در این ساختمان،
ویژگی‌های اصلی یک سیستم هم‌روند 
(مانند امکان رخ دادن دو رویداد بدون وابستگی به یکدیگر و به صورت موازی)
پوشش داده می‌شوند؛ از این جهت، این روش
برای مدل کردن سیستم‌های مورد نظر ما، روش
مناسبی است.

هالپِرن و پِرل ‌در
\cite{halpern2001causes}
یک فریم‌ورک برای تشخیص علت‌
در یک مدل معادله‌ای ارائه داده‌اند.
این فریم‌ورک، بسیاری از ایده‌های طبیعی علّیت را
(مانند در نظر گرفتن سناریوهایی که در دنیای واقعی صورت نپذیرفته‌اند)
پوشش می‌دهد. از طرفی، با توجه به فرم منظم و
استقرایی ساختمان رویداد، ترجمه یک ساختمان رویداد
به مدل معادله‌ای نیز قابل انجام است
(روشی برای ترجمه توسط صیحانی و دیگران در
\cite{seyhani2022}
ارائه شده است).
به این ترتیب، می‌توانیم از فریم‌ورک تشخیص علت
هالپرن و پرل برای استدلال علّی در
یک سیستم هم‌روند استفاده کنیم.

یکی از مسائل اصلی در ترجمه ساختمان رویداد
و تشخیص علت‌، پیچیدگی زمان اجرای الگوریتم‌های
ارائه شده برای این مراحل است. ترجمه ساختمان رویداد
به مدل معادله‌ای، مدلی با اندازه نمایی 
(نسبت به تعداد رویدادها)
نتیجه می‌دهد. هم‌چنین، روش تشخیص علت ارائه شده
در مدل هالپرن و پرل نیز در کلاس مسائل
\lr{$\Sigma_2^P$-Complete}
قرار می‌گیرد. از این جهت، تشخیص علت در
یک ساختمان رویداد، با کمک تعریف اولیه علیت،
تنها روی سیستم‌های بسیار ساده
با تعداد انگشت‌شماری رویداد قابل انجام است.
هدف این پروژه، بهبود زمان اجرای تشخیص علت
برای میسر کردن بررسی سیستم‌های پیچیده‌تر می‌باشد.

نتیجه اصلی این پروژه، پیاده‌سازی ابزار
\lr{EStResT}
\LTRfootnote{
  \url{https://github.com/seyhani/causality/tree/master/estrest}
}
است که امکان توصیف ساختمان رویداد
و بررسی درستی علت واقعی در آن را فراهم می‌کند.
این ابزار با زبان
\lr{Python}
پیاده‌سازی شده است.

\textbf{ساختار گزارش:}
ابتدا در بخش‌های
\ref{sec:event-structure} و \ref{sec:causality}
مفاهیم ساختمان رویداد، مدل علّی
و علّت واقعی را توضیح می‌دهیم.
در بخش
\ref{sec:es-to-cm}
شیوه ترجمه یک ساختمان رویداد
به مدل علی را توضیح می‌دهیم.
سپس در بخش
\ref{sec:improvements}
بهبودهای تئوری مورد نظر برای مساله
علت واقعی را مورد بررسی قرار می‌دهیم.
در بخش
\ref{sec:experiments}
نتایج عملی را می‌آوریم
و در انتها، در بخش
\ref{sec:future}
گام‌های بعدی برای این پژوهش را مطرح می‌کنیم.
