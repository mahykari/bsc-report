\section{ترجمه ساختمان رویداد به مدل علّی}\label{sec:es-to-cm}

برای این‌که بتوانیم از تعریف علیت هالپرن و پرل
برای یک ساختمان رویداد استفاده کنیم، بایستی ابتدا
ساختمان رویداد را به صورت یک مدل علّی بیان کنیم. در ادامه
روشی که توسط صیحانی و دیگران در
\cite{seyhani2022}
برای این ترجمه ارائه شده را توضیح می‌دهیم.

\begin{definition}
  مجموعه $Y$،
  مجموعه $X$
  را \textit{می‌پوشاند}
  اگر و فقط اگر شرط زیر برقرار باشد:
  \begin{equation*}
    X \subset Y \wedge \left(
      \forall Z.\;
      X \subseteq Z \subseteq Y \implies Z=X \vee Z=Y 
    \right)
  \end{equation*}
  رابطه پوشیده شدن
  $X$ با $Y$
  را به صورت
  $X \prec Y$
  نشان می‌دهیم.
\end{definition}

ساختمان رویداد
$\ES=(E,\#,\vdash)$
را در نظر بگیریم، که در آن
$E=\set{e_1,\cdots,e_n}$.
در ادامه مدل علّی
$M=(U,V,F)$
را برای این ساختمان تعریف می‌کنیم.

مجموعه $U$
را برابر مجموعه تهی در نظر می‌گیریم.
برای مجموعه $V$
متغیرهای زیر را در نظر می‌گیریم:
\begin{align}
  V = & \set{C_{e_i,e_j} \mid 1 \leq i < j \leq n} \cup \\
  & \set{
    EN_{S,e} \mid
    S \subseteq E \wedge e \in E \wedge e \not\in S
  } \cup \\
  & \set{
    M_{S,e} \mid
    S \subseteq E \wedge e \in E \wedge e \not\in S
  } \cup \set{PV}
\end{align}

متغیرهای
$C$، $EN$ و $M$
به ترتیب روابط تعارض، فعال‌سازی و فعال‌سازی کمینه
را مدل می‌کنند. همچنین، متغیر
$PV$
برای نشان دادن رفتار ناامن در ساختمان رویداد
(مطابق عبارت \ref{eq:unsafe-behavior})
استفاده می‌شود که در بخش
\ref{sec:improvements}
راجع به آن بیشتر توضیح خواهیم داد.
برای متغیرهای تعارض، داریم:
\begin{equation*}
  F_{C_{e,e'}} = \begin{cases}
    \top  & e \# e' \\
    \bot  & e \cancel{\#} e'
  \end{cases}
\end{equation*}

برای تابع سایر متغیرها، دو تابع کمکی زیر را تعریف می‌کنیم:
\begin{align*}
  \Con(S) & =
  \bigwedge_{e_i,e_j \in S \wedge i < j} \neg C_{e_i,e_j} \\
  \Min(S,e) & = \bigwedge_{
    S' \subseteq E.\;
    (S' \prec S \vee S \prec S') \wedge e \not\in S'
  } \neg M_{S',e}
\end{align*}

به این ترتیب، داریم:
\begin{equation*}
  F_{M_{S,e}} = \begin{cases}
    \Min(S,e) \wedge \Con(S) & S \minen e \\
    \bot                     & S \not\minen e
  \end{cases}
\end{equation*}

\begin{equation*}
  F_{EN_{S,e}} = \left(
    M_{S,e} \vee
    \left( \bigvee_{S' \prec S} EN_{S',e} \right)
  \right) \wedge \Con(S)
\end{equation*}
