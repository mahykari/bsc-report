\section{ساختمان رویداد}\label{sec:event-structure}

روش‌های متفاوتی برای توصیف و مدل کردن
سیستم‌های هم‌روند ارائه شده‌اند.
در توصیف به کمک اتوماتون محدود قطعی
\LTRfootnote{Deterministic finite automaton}
معمولا ترتیب‌های متفاوت اجرای هم‌روند به صورت مستقل
در ساختار مدل توصیف می‌شوند. این روش، ویژگی‌هایی از
یک سیستم هم‌روند واقعی، مانند امکان رخ دادن دو رویداد
به صورت مستقل و بدون وابستگی را فراهم نمی‌کند.
\textit{ساختمان رویداد}
\LTRfootnote{Event structure}
مدلی برای توصیف سیستم‌های هم‌روند است که مشابه
اتوماتون محدود غیرقطعی
\LTRfootnote{Non-deterministic finite automaton}
هم‌روندی رویدادها را به صورت صریح مشخص می‌کند.

\begin{definition}\label{def:event-structure}
  \textit{ساختمان رویداد}
  $\ES$
  یک سه‌تایی به فرم
  $(E, \#, \vdash)$
  است که در آن:
  \begin{enumerate}
    \item $E$
    یک مجموعه از رویداد‌ها است.
    \item $\#$
    رابطه
    \textit{تعارض}
    \LTRfootnote{Conflict}
    است که یک رابطه دودویی متقارن و غیربازتابی بر روی مجموعه
    $E$
    می‌باشد.

    \item $\vdash\;\subseteq \Con(\ES) \times E$
    رابطه
    \textit{فعال‌سازی}
    \LTRfootnote{Enabling}
    است که شرط زیر را برقرار می‌کند:
    \begin{equation*}
        (X \vdash e) \wedge (X \subseteq Y \in \Con(\ES))
        \implies Y \vdash e
    \end{equation*}
  \end{enumerate}
\end{definition}

در رابطه بالا
$\Con$
(مخفف
\lr{Consistent})،
زیرمجموعه‌ای از مجموعه توانی رویدادها است که اعضای آن فاقد تعارض باشند.

به صورت شهودی، رابطه تعارض برای جلوگیری از رخ دادن
دو رویداد در یک اجرا از سیستم،
و رابطه فعال‌سازی برای ایجاد وابستگی بین رویدادها استفاده می‌شود.
برای مشخص کردن مجموعه کمینه وابستگی‌های یک رویداد،
رابطه فعال‌سازی کمینه را تعریف می‌کنیم.

\begin{definition}\label{def:min-enabling}
  به ازای هر ساختمان رویداد، می‌توانیم رابطه
  \textit{فعال‌سازی کمینه}
  را به صورت زیر تعریف کنیم:
  \begin{equation*}
    X \minen e \iff
    (X \vdash e) \wedge
    (\forall Y \subseteq X.\;Y \vdash e \implies Y=X)
  \end{equation*}
\end{definition}

\begin{proposition}
  در هر ساختمان رویداد داریم
  \begin{equation*}
    Y \vdash e \implies
    \exists X \subseteq Y.\;X \minen e
  \end{equation*}
\end{proposition}

برای مشخص کردن وضعیت یک سیستم در هر لحظه
از مفهومی به نام پیکر‌بندی
\LTRfootnote{Configuration}
استفاده می‌شود. یک پیکربندی
مجموعه‌ای شامل رویدادهایی است که تا آن لحظه در سیستم رخ داده‌اند.

\begin{definition}\label{def:configuration}
  اگر
  $\ES = (E,\#,\vdash)$
  یک ساختمان رویداد باشد، یک
  \textit{پیکربندی}
  آن یک زیرمجموعه از رویداد‌ها
  $X \subseteq E$
  است که شرایط زیر را داشته باشد:
  \begin{enumerate}
      \item $X \in \Con(\ES)$
      \item $
        \forall e \in X.\;
        \exists e_0, \cdots, e_n \in X.\;
        e_n = e \wedge
        \forall i \leq n. \set{e_0, \cdots, e_{i-1}} \vdash e_i
      $
  \end{enumerate}
\end{definition}

مجموعه همه پیکربندی‌های یک ساختمان رویداد مانند
$\ES$
با
$\Conf(\ES)$
نمایش داده می‌شود.

\begin{example}\label{ex:event-structure}
  ساختمان رویدادی با مولفه‌های زیر در نظر بگیریم:
  \begin{align*}
    E & = \set{a, b, c} \\
    \# & : a \# b \\
    \vdash & :
      \varnothing \vdash a,\,
      \varnothing \vdash b,\,
      \set{a} \vdash c,\,
      \set{b} \vdash c
  \end{align*}

  شبکه
  \LTRfootnote{Lattice}
  پیکربندی‌های این ساختمان در شکل
  \ref{fig:es-example-configs}
  آمده است.

  \begin{figure}
  \centering
  \def\W{2}
  \def\H{1.25}
  \begin{tikzpicture}
      \node (null) at (0,0) {$\varnothing$};
      \node (a)  at (-\W,  \H) {$\set{a}$};
      \node (b)  at ( \W,  \H) {$\set{b}$};
      \node (ac) at (-\W,2*\H) {$\set{a,c}$};
      \node (bc) at ( \W,2*\H) {$\set{b,c}$};
      \draw[->] (null) -- (a);
      \draw[->] (null) -- (b);
      \draw[->] (a)    -- (ac);
      \draw[->] (b)    -- (bc);
  \end{tikzpicture}
  \caption{
    شبکه پیکربندی‌های مثال
    \ref{ex:event-structure}
  }
  \label{fig:es-example-configs}
\end{figure}

\end{example}