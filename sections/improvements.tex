\section{بهبود پیچیدگی زمانی استدلال علّی}\label{sec:improvements}

\subsection{علت واقعی تک‌متغیره}

\subsection{کاهش اندازه مدل علّی}

دوتا از منابع پیچیدگی در مساله تصمیم علیت،
به دلیل جستجو در مجموعه توانی متغیرهای مدل علّی هستند.
به همین دلیل، می‌توانیم تلاش کنیم تا اندازه مدل علّی را
تا حد ممکن کوچک کنیم. مفهوم
\textit{تصویر}
مدل علّی که توسط هاپکینز در
\cite{hopkins2002strategies}
گامی در جهت حذف تعدادی از متغیرها در
یک نمونه
\LTRfootnote{Instance}
از مساله تصمیم علیت است.

\begin{definition}\label{def:w-projection}
  فرض کنیم
  $M=(U,V,F)$
  یک مدل علّی و
  $(M,u)$
  یک دنیای علّی است
  $(V={V_1,\cdots,V_n})$.
  \textit{حذف کردن}
  متغیر
  $V_i$
  از
  $(M,u)$
  را به صورت زیر تعریف می‌کنیم:
  \begin{enumerate}[label=(\alph*)]
    \item $V_i$
    را از
    $V$
    حذف می‌کنیم.
    \item
    به ازای هر فرزند
    $V_i$
    مانند
    $X$،
    در
    $F_X$
    به‌جای
    $V_i$،
    مقدار
    $v_i=V_i(u)$
    را جایگزین می‌کنیم.
  \end{enumerate}
  
  \textit{تصویر}\LTRfootnote{Projection} $(M,u)$
  روی متغیرهای
  $V_1,\cdots,V_k$
  را به صورت یک مدل علّی جدید تعریف می‌کنیم
  که در آن، متغیرهای
  $V_{k+1},\cdots,V_n$
  از
  $(M,u)$
  حذف شده‌اند.
  به همین ترتیب،
  برای
  $(M,u)$
  \textit{تصویر $W$}\LTRfootnote{W-projection}
  به‌ازای مساله علیت
  $X$ و $Y$
  را به صورت تصویر
  $(M,u)$
  روی متغیرهای زیر تعریف می‌کنیم:
  \begin{enumerate}[label=(\alph*)]
    \item $X,Y$
    \item هر راسی مانند
    $V^{XY}$
    که در شبکه علّی مورد نظر،
    روی مسیری از
    $X$ به $Y$
    قرار دارد.
    \item $\Par(V)$
    به ازای هر
    $V$
    که در مجموعه
    $V^{XY}$ها
    و $Y$
    قرار دارد.
  \end{enumerate}
\end{definition}

\begin{theorem}\label{th:conf-equiv}
  با داشتن یک ساختمان رویداد
  $\ES=(E,\#,\vdash)$
  و مجموعه
  $S = \set{e_1,\cdots,e_n} \subseteq E$،
  داریم
  \begin{equation*}
    S \in \Conf(\ES) \iff
    S \in \Con(E) \wedge
    \exists \pi \in \Pi(S).\;
    \varnothing \vdash \pi_1, \cdots,
    \set{\pi_1,\cdots,\pi_{n-1}} \vdash \pi_n
  \end{equation*}
  که
  $\Pi(S)$
  مجموعه همه جایشگت‌های رویدادهای
  $S$
  است.
\end{theorem}

\begin{proof}
  این قضیه را در هر دو جهت اثبات می‌کنیم.
  \begin{itemize}
    \item $\implies$:
    این گزاره به صورت مستقیم از تعریف
    \ref{def:configuration}
    نتیجه می‌شود.
    \item $\impliedby$:
    از فرض مساله می‌توان نتیجه گرفت که حداقل
    یک مسیر در شبکه پیکربندی‌های
    $\ES$
    از مجموعه تهی
    ($\varnothing$)
    تا $S$
    وجود دارد؛ بنابر این،
    $S$
    یک پیکربندی
    $\ES$
    است و حکم اثبات می‌شود.
  \end{itemize}
\end{proof}