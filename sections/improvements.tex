\section{بهبود پیچیدگی زمانی استدلال علّی}\label{sec:improvements}

\subsection{علت واقعی تک‌متغیره}

\subsection{کاهش اندازه مدل علّی}

دوتا از منابع پیچیدگی در مساله تصمیم علیت،
به دلیل جستجو در مجموعه توانی متغیرهای مدل علّی هستند.
به همین دلیل، می‌توانیم تلاش کنیم تا اندازه مدل علّی را
تا حد ممکن کوچک کنیم. مفهوم
\textit{تصویر}
مدل علّی که توسط هاپکینز در
\cite{hopkins2002strategies}
گامی در جهت حذف تعدادی از متغیرها در
یک نمونه
\LTRfootnote{Instance}
از مساله تصمیم علیت است.

\begin{definition}\label{def:w-projection}
  فرض کنیم
  $M=(U,V,F)$
  یک مدل علّی و
  $(M,u)$
  یک دنیای علّی است
  $(V={V_1,\cdots,V_n})$.
  { \color{blue}
    حذف و تصویر رو اینجا باید توضیح بدم.
  }
\end{definition}

\begin{theorem}\label{th:conf-equiv}
  با داشتن یک ساختمان رویداد
  $\ES=(E,\#,\vdash)$
  و مجموعه
  $S = \set{e_1,\cdots,e_n} \subseteq E$،
  داریم
  \begin{equation*}
    S \in \Conf(\ES) \iff
    S \in \Con(E) \wedge
    \exists \pi \in \Pi(S).\;
    \varnothing \vdash \pi_1, \cdots,
    \set{\pi_1,\cdots,\pi_{n-1}} \vdash \pi_n
  \end{equation*}
  که
  $\Pi(S)$
  مجموعه همه جایشگت‌های رویدادهای
  $S$
  است.
\end{theorem}

\begin{proof}
  این قضیه را در هر دو جهت اثبات می‌کنیم.
  \begin{itemize}
    \item[$\implies$:]
    این گزاره به صورت مستقیم از تعریف
    \ref{def:configuration}
    نتیجه می‌شود.
    \item[$\impliedby$:]
    از فرض مساله می‌توان نتیجه گرفت که حداقل
    یک مسیر در شبکه پیکربندی‌های
    $\ES$
    از مجموعه تهی
    ($\varnothing$)
    تا $S$
    وجود دارد؛ بنابر این،
    $S$
    یک پیکربندی
    $\ES$
    است و حکم اثبات می‌شود.
  \end{itemize}
\end{proof}