\section{بهبود پیچیدگی زمانی استدلال علّی}\label{sec:improvements}

\subsection{علت واقعی تک‌متغیره}
با توجه به تعریف مساله علت واقعی،
برای علت‌های چندمتغیره، بررسی شرط کمینه بودن
اندازه علت معادل حل کردن مساله علت واقعی به ازای
همه زیرمجموعه‌های علت ارائه شده می‌باشد.
بنابراین، یکی از راه‌های کاهش پیچیدگی مساله علت واقعی،
ساده‌تر شدن بررسی شرط کمینه بودن است.
قضیه زیر که توسط آیتر و لوکاسویچ در
\cite{eiter2001complexity}
ارائه شده است، به ما در ساده کردن
بررسی شرط کمینه بودن کمک می‌کند.

\begin{theorem}\label{thm:causality-singleton}
  با داشتن
  $(M,u)$،
  $X$ و $Y$،
  اگر
  $X=x$
  علت واقعی
  $Y=y$ باشد،
  $X$
  یک تک‌متغیر است.
\end{theorem}

\begin{proof}[اثبات اولیه]
  با استفاده از برهان خلف و نشان دادن این‌که همیشه
  می‌توانیم همه متغیرهای
  $X$
  (بجز یک متغیر)
  را به مجموعه
  $W$
  منتقل کنیم، این قضیه ثابت خواهد شد.
  اثبات کاملتر توسط آیتر و لوکاسویچ ارائه شده است.
\end{proof}

به کمک این قضیه، می‌توانیم بررسی شرط کمینه بودن را
با پیچیدگی زمانی ثابت انجام دهیم.
توجه کنیم که این قضیه به تنهایی پیچیدگی کلی
مساله علیت را کاهش نمی‌دهد و همچنان این مساله در کلاس
\lr{$\Sigma_2^P$-Complete}
قرار دارد.

\subsection{کاهش اندازه مدل علّی}

دوتا از منابع پیچیدگی در مساله تصمیم علیت،
به دلیل جستجو در مجموعه توانی متغیرهای مدل علّی هستند.
به همین دلیل، می‌توانیم تلاش کنیم تا اندازه مدل علّی را
تا حد ممکن کوچک کنیم. مفهوم
\textit{تصویر}
مدل علّی که توسط هاپکینز در
\cite{hopkins2002strategies}
ارائه شده است،
گامی در جهت حذف تعدادی از متغیرها در
یک نمونه
\LTRfootnote{Instance}
از مساله تصمیم علیت است.

\begin{definition}\label{def:variable-deletion}
  فرض کنیم
  $M=(U,V,F)$
  یک مدل علّی و
  $(M,u)$
  یک دنیای علّی است
  $(V={V_1,\cdots,V_n})$.
  \textit{حذف کردن}
  متغیر
  $V_i$
  از
  $(M,u)$
  را به صورت زیر تعریف می‌کنیم:
  \begin{enumerate}[label=(\alph*)]
    \item $V_i$
    را از
    $V$
    حذف می‌کنیم.
    \item
    به ازای هر فرزند
    $V_i$
    مانند
    $X$،
    در
    $F_X$
    به‌جای
    $V_i$،
    مقدار
    $v_i=V_i(u)$
    را جایگزین می‌کنیم.
  \end{enumerate}
\end{definition}

به صورت شهودی، حذف کردن یک متغیر مانند
$V$
از مدل علّی، معادل ثابت نگه داشتن مقدار آن
(با مقداردهی $V(u)$)
می‌باشد.

\begin{definition}\label{def:projection}
  \textit{تصویر}\LTRfootnote{Projection} $(M,u)$
  روی متغیرهای
  $V_1,\cdots,V_k$
  را به صورت یک مدل علّی جدید تعریف می‌کنیم
  که در آن، متغیرهای
  $V_{k+1},\cdots,V_n$
  از
  $(M,u)$
  حذف شده‌اند.
\end{definition}

\begin{definition}\label{def:w-projection}
  برای
  $(M,u)$
  \textit{تصویر $W$}\LTRfootnote{W-projection}
  به‌ازای مساله علیت
  $X$ و $Y$
  را به صورت تصویر
  $(M,u)$
  روی متغیرهای زیر تعریف می‌کنیم:
  \begin{enumerate}[label=(\alph*)]
    \item $X,Y$
    \item هر راسی مانند
    $V^{XY}$
    که در شبکه علّی مورد نظر،
    روی مسیری از
    $X$ به $Y$
    قرار دارد.
    \item $\Par(V)$
    به ازای هر
    $V$
    که در مجموعه
    $V^{XY}$ها
    و $Y$
    قرار دارد.
  \end{enumerate}
\end{definition}

در ادامه می‌توانیم نشان دهیم که
برای حل کردن مساله تصمیم علیت برای
$X$ و $Y$،
تنها کافی‌است متغیرهایی را در نظر بگیریم که
روی مسیری از
$X$ به $Y$
قرار دارند، یا روی مقدار این متغیرها
تاثیر مستقیم دارند.
قضیه زیر که توسط هاپکینز ارائه شده است،
از مفهوم
تصویر $W$
برای حل مساله علیت استفاده می‌کند.

\begin{theorem}\label{thm:causality-w-projection}
  فرض کنیم
  $M=(U,V,F)$
  یک مدل علّی و
  $(M,u)$
  یک دنیای علّی است.
  $X=x$
  علت واقعی
  $Y=y$
  در
  $(M,u)$
  است، اگر و فقط اگر در
  $(M',u)$
  نیز علت واقعی باشد و
  $M'$
  هم
  تصویر $W$ی
  دنیای علی
  $(M,u)$
  (نسبت به
  $X$ و $Y$)
  باشد.
\end{theorem}

به این ترتیب، حل کردن مساله علیت در مدلی که
به کمک
تصویر $W$
کوچک شده، معادل حل کردن آن در مدل اولیه است.
کاهش اندازه مدل
(و به دست آوردن تصویر $W$)
در زمان چندجمله‌ای قابل انجام است،
و زمان مورد نیاز برای حل مساله علیت در مدل کوچک‌شده
نیز به صورت  قابل توجهی
(با توجه به پیچیدگی مساله)
کاهش می‌یابد.

\textbf{تعریف متغیر اثر در رفتار ناامن ساختمان رویداد:}
برای استفاده از تکنیک
تصویر $W$
در مدل علّی ساختمان رویداد،
بایستی متغیری را به عنوان متغیر اثر تعریف کنیم.
درستی این متغیر، معادل
امکان برقراری حداقل یکی از پیکربندی‌های ناامن می‌باشد.
در قدم اول، رابطه‌ای تعریف می‌کنیم که معادل
امکان رخ دادن یک پیکربندی در ساختمان رویداد است.

\begin{theorem}\label{th:conf-equiv}
  با داشتن یک ساختمان رویداد
  $\ES=(E,\#,\vdash)$
  و مجموعه
  $S = \set{e_1,\cdots,e_n} \subseteq E$،
  داریم
  \begin{equation}\label{eq:conf-equiv}
    S \in \Conf(\ES) \iff
    S \in \Con(E)
    \wedge \exists \pi \in \Pi(S).\;
    \varnothing \vdash \pi_1 \wedge \cdots \wedge
    \set{\pi_1,\cdots,\pi_{n-1}} \vdash \pi_n
  \end{equation}
  که
  $\Pi(S)$
  مجموعه همه جایشگت‌های رویدادهای
  $S$
  است.
\end{theorem}

\begin{proof}
  این قضیه را در هر دو جهت اثبات می‌کنیم.
  \begin{itemize}
    \item $\implies$:
    این گزاره به صورت مستقیم از تعریف
    \ref{def:configuration}
    نتیجه می‌شود.
    \item $\impliedby$:
    از فرض مساله می‌توان نتیجه گرفت که حداقل
    یک مسیر در شبکه پیکربندی‌های
    $\ES$
    از مجموعه تهی
    ($\varnothing$)
    تا $S$
    وجود دارد؛ بنابر این،
    $S$
    یک پیکربندی
    $\ES$
    است و حکم اثبات می‌شود.
  \end{itemize}
\end{proof}

به این ترتیب، می‌توانیم از عبارتی معادل
امکان رخ دادن یک پیکربندی در مدل علّی ساختمان رویداد
استفاده کنیم. توجه کنیم که گزاره
\ref{eq:conf-equiv}
معادل گزاره زیر است که با متغیرهای مدل علّی
ساختمان رویداد بیان شده است:
\begin{equation}\label{eq:conf-equiv-cm}
  S \in \Conf(\ES) \iff
  \Con(S) \wedge
  \bigvee_{\pi \in \Pi(S)} \left(
    EN_{\varnothing, \pi_1} \wedge \cdots \wedge
    EN_{\set{\pi_1,\cdots,\pi_{n-1}}, \pi_n}
  \right)
\end{equation}

به این ترتیب، توصیف رفتار ناامن نیز به صورت زیر قابل انجام است:

\begin{multline}\label{eq:unsafe-behavior-equiv-cm}
  \exists S_i \in S.\;S_i \in \Conf(\ES) \iff \\
  \bigvee_{S_i \in S} \left(
    \Con(S_i) \wedge
    \bigvee_{\pi \in \Pi(S_i)} \left(
      EN_{\varnothing, \pi_1} \wedge \cdots \wedge
      EN_{\set{\pi_1,\cdots,\pi_{n-1}}, \pi_n}
    \right)
  \right)
\end{multline}

\subsection{کاهش پیچیدگی بررسی $Z$ها}
هاپکینز در
\cite{hopkins2002strategies}
روشی برای کاهش تعداد زیرمجموعه‌های
$Z$
که برای بخش
\lr{AC2.c}
مساله علت واقعی باید بررسی شوند، ارائه داده است.
این روش به ازای هر مدل علّی قابل استفاده است.
اما می‌دانیم که مدل علّی مورد استفاده ما برای ساختمان رویداد،
یک مدل دودویی
\LTRfootnote{Binary}
است. در مدل دودویی، بررسی شرط
\lr{AC2.c}
به صورت ساده‌تری قابل انجام است.

قضیه زیر توسط آیتر و لوکاسویچ در
\cite{eiter2001complexity}
ارائه شده است.

\begin{theorem}
  به‌ازای دنیای علّی دودویی
  $(M,u)$،
  $X=x$
  یک علت ضعیف برای
  $Y=y$
  است اگر و فقط اگر
  شرط
  \lr{AC1}
  برقرار باشد،
  مجموعه
  $W \subseteq V \setminus X$
  وجود داشته باشد که
  \lr{AC2.a} و \lr{AC2.b}
  برقرار باشند، و شرط
  \lr{AC2.c'}
  نیز طبق تعریف زیر برقرار باشد:
  
  \textbf{\lr{AC2.c'}}
    برای
    $Z=V \setminus (X \cup W)$، داریم
    $Z_{xw}(u)=Z(u)$.
\end{theorem}

\begin{proof}[اثبات مقدماتی]\label{thm:causality-z-eiter}
  ایده اثبات این قضیه نیز مشابه قضیه
  \ref{thm:causality-singleton} است.
  اثبات قضیه با نشان دادن این‌که می‌توانیم متغیرهایی از
  $Z$
  را به گونه‌ای به
  $W$
  منتقل کنیم که شرط جدید نیز برقرار شود، انجام می‌شود.
\end{proof}

هاپکینز در
\cite{hopkins2002strategies}
قضیه زیر را بر اساس قضیه
\ref{thm:causality-z-eiter}
و برای
تصویر $W$
مدل‌های دودویی ارائه می‌کند:

\begin{theorem}\label{thm:causality-z-hopkins}
  فرض کنیم
  $M$
  یک مدل علّی تصویر شده و دودویی باشد.
  فرض کنیم به ازای
  $(M,u)$ و $X,Y,W$،
  شرط
  \lr{AC1} و دو بخش اول شرط \lr{AC2}
  برقرارند. بخش سوم شرط
  \lr{AC2}
  برقرار است اگر و تنها اگر به‌ازای
  $Z=V \setminus (X \cup W)$ و $z=Z(u)$
  داشته باشیم
  $Y_{xwz}(u)=y$.
\end{theorem}

به این ترتیب، به کمک قضیه‌های
\ref{thm:causality-singleton} و \ref{thm:causality-z-eiter}
می‌توانیم نشان دهیم که
مساله علت واقعی برای مدل‌های دودویی در کلاس
\lr{NP-Complete}
قرار می‌گیرد. همچنین، با فرض این‌که
$W$ و $w$
برای بررسی شرط
\lr{AC2}
به ما داده شده است، مساله علت واقعی
با پیچیدگی چندجمله‌ای قابل حل است.